\begin{frame}{问题建议:个人层面}
    \begin{exampleblock}{价值观}
        \begin{itemize}
            \item \textbf{追求差异化}:不满足于平庸,重视技术深度,致力于在工作中体现独特的个人价值。
            \item \textbf{兴趣驱动}:兴趣与效率强绑定,对于感兴趣的技术方向,愿意投入超额的时间与精力。
            \item Demand-oriented and practical; First-hand or global; Not satisfied.
        \end{itemize}
    \end{exampleblock}

    \begin{alertblock}{知识的诅咒}
        \begin{itemize}
            \item 在教学与沟通中,需时刻警惕“知识的诅咒”,避免因预设听众具备同等背景知识而导致沟通障碍,需加强换位思考能力。
        \end{itemize}
    \end{alertblock}
\end{frame}

\begin{frame}[allowframebreaks]{问题建议:部门层面}
    \begin{block}{技术资源}
        \begin{itemize}
            \item \textbf{技术审美}:建议利用空余时间增加内部技术交流,提升整体的技术品味与视野。
            \item \textbf{硬件资源}:希望能提供 API Key 或显卡硬件支持,以加速大模型课程开发与实验。
        \end{itemize}
    \end{block}

    \begin{alertblock}{标准化}
        \begin{itemize}
            \item \textbf{课程开发标准}:目前缺乏硬性的标准化规范,导致不同课程间风格、质量与组织方式各不相同,增加了后续维护成本。
        \end{itemize}
    \end{alertblock}

    \begin{alertblock}{数据安全}
        在下载 Ai-Lab 作业时,没有批量下载功能,于是进行抓包分析,发现全部文件存储在\href{https://dq-resource-online.obs.cn-south-1.myhuaweicloud.com}{华为对象存储}中,且未限制目录索引,允许匿名遍历所有文件,经测试共发现 34,268 个去重文件,共 80.3 GB。此外在点宽杯提交作品中发现了未失效的 DeepSeek API key。

        \begin{center}
            \begin{tabular}{llllllll}
                \toprule
                \textbf{.mp4} & \textbf{.csv} & \textbf{.tar} & \textbf{.pptx} & \textbf{.zip} & \textbf{.exe} & \textbf{.json} & \textbf{.pdf} \\
                \midrule
                25.8 GB & 18.1 GB & 11.5 GB & 9.6 GB & 4.7 GB & 2.7 GB & 2.4 GB & 1.4 GB \\
                \bottomrule
            \end{tabular}
        \end{center}
    \end{alertblock}

    \begin{alertblock}{带宽成本 by AI}
        当前系统使用对象存储直接传输频繁访问的课程代码案例等静态资源,但对象存储的流量费用较高,通常按下载流量计费,无缓存优化。根据访问模式分析,资源多为重复下载内容,若采用 CDN 服务,可通过边缘缓存减少回源流量,从而显著降低带宽成本。
    \end{alertblock}
\end{frame}
