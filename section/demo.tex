\subsection{基础文本与列表}
\begin{frame}{文本样式与列表}
    \begin{columns}
        \begin{column}{0.5\textwidth}
            \textbf{文本样式}
            \begin{itemize}
                \item 普通文本
                \item \textbf{粗体文本}
                \item \textit{斜体文本}
                \item \alert{强调文本 (Alert)}
                \item \textcolor{main}{主色调文本}
            \end{itemize}
        \end{column}
        \begin{column}{0.5\textwidth}
            \textbf{列表环境}
            \begin{enumerate}
                \item 第一项
                \item 第二项
                    \begin{itemize}
                        \item 子项 A
                        \item 子项 B
                    \end{itemize}
                \item 第三项
            \end{enumerate}
        \end{column}
    \end{columns}
\end{frame}

\begin{frame}{描述列表}
    \begin{description}
        \item[API] 应用程序接口 \pause
        \item[GUI] 图形用户界面 \pause
        \item[CLI] 命令行界面
    \end{description}

    \vspace{1em}
    \textbf{覆盖示例:}
    \begin{itemize}
        \item<3-> 第三步出现
        \item<4-> 第四步出现
        \item<5-> 第五步出现
    \end{itemize}
\end{frame}

\subsection{区块与数学公式}
\begin{frame}{Beamer 区块}
    \begin{block}{普通区块}
        这是一个标准的 Beamer 区块,用于放置一般信息。
    \end{block}

    \begin{alertblock}{警告区块}
        这是一个警告区块,用于强调重要或危险的信息。
    \end{alertblock}

    \begin{exampleblock}{示例区块}
        这是一个示例区块,常用于展示案例或证明。
    \end{exampleblock}
\end{frame}

\begin{frame}{数学公式}
    行内公式:$E = mc^2$ 或 $e^{i\pi} + 1 = 0$

    \vspace{1em}
    行间公式(积分):
    \[
        \int_{-\infty}^{\infty} e^{-x^2} \, dx = \sqrt{\pi}
    \]

    矩阵示例:
    \[
        A = \begin{pmatrix}
            1 & 0 \\
            0 & 1
        \end{pmatrix}, \quad
        B = \begin{bmatrix}
            x & y \\
            z & w
        \end{bmatrix}
    \]
\end{frame}

\subsection{代码展示}
\begin{frame}[fragile]{Python 代码环境 (ipyenv)}
    使用 \texttt{ipyenv} 环境展示 Python 代码:
    \begin{ipyenv}{数据处理脚本}
import pandas as pd
import numpy as np

def process_data(df):
    """处理数据框"""
    # 计算移动平均
    df['ma_5'] = df['close'].rolling(5).mean()
    return df

# 示例调用
df = pd.DataFrame({'close': np.random.randn(100)})
result = process_data(df)
print(result.head())
    \end{ipyenv}

    行内代码示例:使用 \ipycmd{pip install pandas} 安装库。
\end{frame}

\begin{frame}[fragile]{Shell 代码环境}
    使用 \texttt{ishenv} 环境展示 Bash/Shell 命令:
    \begin{ishenv}{服务器配置}
#!/bin/bash
# 更新系统
sudo apt update && sudo apt upgrade -y

# 安装 Nginx
sudo apt install nginx

# 启动服务
sudo systemctl start nginx
echo "Nginx started successfully"
    \end{ishenv}

    行内命令示例:运行 \ishcmd{ls -la} 查看文件列表。
\end{frame}

\subsection{表格与图片}
\begin{frame}{三线表}
    \begin{table}
        \centering
        \caption{编程语言流行度示例}
        \begin{tabular}{llrr}
            \toprule
            \textbf{排名} & \textbf{语言} & \textbf{份额 (\%)} & \textbf{变化} \\
            \midrule
            1 & Python & 25.4 & +2.1 \\
            2 & C++    & 18.2 & -0.5 \\
            3 & Java   & 14.8 & -1.2 \\
            4 & Go     & 9.5  & +3.4 \\
            \bottomrule
        \end{tabular}
    \end{table}
\end{frame}

\begin{frame}{图片与图文混排}
    \begin{columns}
        \begin{column}{0.6\textwidth}
            \begin{itemize}
                \item 左侧是文本说明
                \item 右侧是图片展示
                \item 图片可以是 Logo 或示意图
                \item 使用 \texttt{columns} 环境实现分栏
            \end{itemize}
        \end{column}
        \begin{column}{0.4\textwidth}
            \centering
            \includegraphics[width=0.8\textwidth]{static/image/logo.png}
        \end{column}
    \end{columns}
\end{frame}
