\section{基础文本与列表}
\begin{frame}{基础文本与列表布局}
    \begin{columns}[T]
        \begin{column}{0.33\textwidth}
            \textbf{\color{main} 字体与颜色}
            \begin{itemize}
                \item \textbf{粗体 Bold} \quad \textit{斜体 Italic}
                \item \texttt{等宽 Monospace}
                \item \alert{警告 Alert} \quad \textcolor{main}{主色 Main}
                \item 数学: $\mathcal{A} \mathfrak{B} \mathbb{C}$
                \item 符号: $\to \implies \infty$
            \end{itemize}
        \end{column}
        \begin{column}{0.33\textwidth}
            \textbf{\color{main} 列表环境}
            \begin{enumerate}
                \item 有序列表项 1
                \item 有序列表项 2
                \begin{itemize}
                    \item 无序子项 A
                    \item[-] 自定义符号
                \end{itemize}
            \end{enumerate}
        \end{column}
        \begin{column}{0.33\textwidth}
            \textbf{\color{main} 描述列表}
            \begin{description}
                \item[TikZ] 绘图包
                \item[Minted] 代码高亮
                \item[Beamer] 演示文稿
            \end{description}
        \end{column}
    \end{columns}
\end{frame}

\section{区块与数学公式}
\begin{frame}{区块封装与数学公式}
    \begin{columns}[T]
        \begin{column}{0.48\textwidth}
            \begin{block}{标准区块 (Block)}
                用于定义、定理或一般说明。
            \end{block}
            \begin{alertblock}{警告区块 (Alert Block)}
                用于强调关键点或注意事项。
            \end{alertblock}
            \begin{exampleblock}{示例区块 (Example Block)}
                $E=mc^2$ 是最著名的公式。
            \end{exampleblock}
        \end{column}
        \begin{column}{0.48\textwidth}
            \textbf{\color{main} 数学排版 (physics 包)}
            \begin{itemize}
                \item 导数: $\dv{f}{x}, \pdv{f}{x, y}$
                \item 矩阵: $\mqty(a & b \\ c & d)$
                \item 括号: $\qty(\frac{a}{b}), \ev{H}$
            \end{itemize}
            \textbf{多行公式:}
            \begin{equation}
                \begin{aligned}
                    \mathcal{L} &= - \sum_{i=1}^N y_i \log(\hat{y}_i) \\
                                &= \text{Cross Entropy}
                \end{aligned}
            \end{equation}
        \end{column}
    \end{columns}
\end{frame}

\section{代码与表格}
\begin{frame}[fragile]{代码高亮与三线表}
    \begin{columns}[T]
        \begin{column}{0.55\textwidth}
            \textbf{\color{main} Python 代码 (ipyenv)}
            \begin{ipyenv}{Data Processing}
import pandas as pd
# 链式操作示例
df = (pd.read_csv('data.csv')
      .query('close > 100')
      .assign(ret=lambda x: x.close.pct_change()))
            \end{ipyenv}
            \vspace{0.5em}
            \textbf{Shell 命令 (ishenv)}
            \begin{ishenv}{Server Setup}
# 部署脚本
pip install -r requirements.txt
gunicorn -w 4 -b 0.0.0.0:8000 app:server
            \end{ishenv}
        \end{column}
        \begin{column}{0.42\textwidth}
            \textbf{\color{main} 经典三线表 (Booktabs)}
            \begin{table}
                \centering
                \small
                \begin{tabular}{llr}
                    \toprule
                    \textbf{模型} & \textbf{参数} & \textbf{精度} \\
                    \midrule
                    ResNet-18 & 11M & 89.2\% \\
                    ViT-Base  & 86M & 94.5\% \\
                    \bottomrule
                \end{tabular}
            \end{table}
            \vspace{1em}
            \textbf{行内代码:} \\
            运行 \ishcmd{git status} 查看状态。
        \end{column}
    \end{columns}
\end{frame}

\section{绘图示例}
\begin{frame}{TikZ 基础绘图示例}
    \centering
    % 引入必要的 TikZ 库
    \usetikzlibrary{positioning, arrows.meta}
    \begin{tikzpicture}[
        node distance=2cm,
        % 定义样式:方框、圆形、箭头
        box/.style={rectangle, draw=main, fill=main!10, thick, minimum size=1cm, rounded corners},
        circ/.style={circle, draw=alert, fill=alert!10, thick, minimum size=0.8cm},
        arrow/.style={-Stealth, thick, color=main}
    ]
        % 1. 放置节点
        \node[box] (input) {输入数据};
        \node[box, right=of input] (process) {处理模块};
        \node[circ, right=of process] (output) {输出};
        % 2. 绘制连线
        \draw[arrow] (input) -- node[above, font=\tiny]{raw} (process);
        \draw[arrow] (process) -- node[above, font=\tiny]{clean} (output);
        % 3. 添加说明
        \node[below=0.5cm of process, color=gray, font=\footnotesize] {简单的线性流程图};
    \end{tikzpicture}
\end{frame}
