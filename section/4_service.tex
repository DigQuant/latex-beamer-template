\begin{frame}{服务工作:效率提升与质量保障}
    \begin{block}{点宽杯作品评审 (12h)}
        \begin{itemize}
            \item \textbf{创新工具}:研发基于 ELO 评分的 AI 作业评分系统 (EloGrader)。
            \item \textbf{成效}:处理 178 份作业,输出可解释排名;\alert{节省人工工时约 15 小时}。
        \end{itemize}
    \end{block}

    \begin{block}{日常教学服务 (28h)}
        \begin{itemize}
            \item \textbf{师资培训核查}:双轮核查共 24h,确保交付质量。
            \item \textbf{毕业论文指导}:投入 4h (集体+个体);优化辅助工具 ThesisTopicAssistant 与 KaggleRag。
        \end{itemize}
    \end{block}
\end{frame}

\begin{frame}{产品工作:工具探索与资产保护}
    \begin{block}{Ai-Lab 授课管理探索 (4h)}
        \begin{itemize}
            \item 全方位探索课程、工具与教学流程的数字化管理模式。
        \end{itemize}
    \end{block}

    \begin{exampleblock}{资产备份系统:Crawler4Das (8h)}
        \begin{itemize}
            \item \textbf{背景}:为课程梳理提供数据安全保障。
            \item \textbf{功能}:实现 Das 平台的全量自动增量备份。
            \item \textbf{覆盖范围}:全部课件、讲义、代码案例与数据集,防止核心资产丢失。
        \end{itemize}
    \end{exampleblock}
\end{frame}
