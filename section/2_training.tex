\begin{frame}{内蒙古财经大学职业学院《AIGC 赋能课程》}
    \begin{block}{工作投入}
        \begin{itemize}
            \item \textbf{备课}:32 小时
            \item \textbf{线上授课}:4 天共 8 小时
        \end{itemize}
    \end{block}

    \begin{exampleblock}{产出成果:大模型相关案例}
        \begin{itemize}
            \item 毕业论文选题辅助 (ThesisTopicAssistant)
            \item AI 生成文本检测器 (AiGeneratedTextDetector)
        \end{itemize}
    \end{exampleblock}
\end{frame}

\begin{frame}[allowframebreaks]{珠海科技学院《Python 商务大数据分析实训》}
    \begin{block}{工作投入}
        \begin{itemize}
            \item \textbf{备课}:16 小时
            \item \textbf{线下授课}:2 天共 16 小时
        \end{itemize}
    \end{block}

    \begin{exampleblock}{产出成果:数据、案例}
        \begin{itemize}
            \item \textbf{数据集}:1500 条 BOSS 直聘岗位信息 / 5991 条淘宝商品评价
            \item \textbf{实战案例}:招聘岗位数据清洗统计 (JobSkillMining)
        \end{itemize}
    \end{exampleblock}

    \begin{alertblock}{问题反馈}
        \begin{itemize}
            \item \textbf{课前调查}:对学生基础的评估维度单调,“有无”选项无法精准反映班级基础层次。
            \item \textbf{线上体验}:实训需要经常在机房内走动,帮学生解决具体问题,线上体验感较差。
            \item \textbf{作业收集}:课后作业收集表权限需手动设置,依赖群内沟通,工具效率待优化。
        \end{itemize}
    \end{alertblock}
\end{frame}

\begin{frame}[allowframebreaks]{山西大学《大数据大模型应用部署工程师实训》}
    \begin{block}{工作投入}
        \begin{itemize}
            \item \textbf{备课}:88 小时
            \item \textbf{授课}:线下 5 天 32 小时 + 线上答辩 3 小时
        \end{itemize}
    \end{block}

    \begin{exampleblock}{产出成果:大模型相关案例}
        \begin{itemize}
            \item \textbf{文本生成}:特定歌手风格歌词生成 (LyricLstm, LyricTransformer)
            \item \textbf{数据分析}:大模型岗位技能聚类分析 (JobSkillEmbeddingAnalysis)
            \item \textbf{应用开发}:本地文档智能问答 (LocalDocRag)
            \item \textbf{智能Agent}:AI 智能阅卷评分系统 (AutoGradingAgent)
            \item \textbf{多模态}:音色克隆 (VoiceCloning)
        \end{itemize}
    \end{exampleblock}

    \begin{alertblock}{问题反馈}
        \begin{itemize}
            \item \textbf{课前调查}:对学生基础的评估维度单调,“有无”选项无法精准反映班级基础层次。
            \item \textbf{环境差异}:学生电脑会出现问题:系统文件缺失、驱动过时、空间不足、代理异常。
            \item \textbf{作业收集}:课后作业收集表权限需手动设置,依赖群内沟通,工具效率待优化。
        \end{itemize}
    \end{alertblock}
\end{frame}

\begin{frame}{师资培训《AI 赋能教学科研实战工作坊》}
    \begin{block}{工作投入}
        \begin{itemize}
            \item \textbf{备课}:32 小时
            \item \textbf{线下授课}:线下 2 天 8 小时
        \end{itemize}
    \end{block}

    \begin{exampleblock}{产出成果}
        \begin{itemize}
            \item \textbf{数据探索}:智能数据探索工具 (DataFormulator)
            \item \textbf{数据检索}:Kaggle 数据集与代码笔记的语义检索工具 (KaggleRag)
        \end{itemize}
    \end{exampleblock}

    \begin{alertblock}{问题反馈}
        \begin{itemize}
            \item \textbf{课前调查}:部分老师没有代码基础,备课内容中途推翻。
        \end{itemize}
    \end{alertblock}
\end{frame}

\begin{frame}{韩山师范学院《人工智能金融应用实战》}
    \begin{block}{工作投入}
        \begin{itemize}
            \item \textbf{备课}:4 小时
            \item \textbf{线下授课}:线上 2 天 16 小时
        \end{itemize}
    \end{block}

    \begin{alertblock}{问题反馈}
        \begin{itemize}
            \item \textbf{课前调查}:对学生基础的评估维度单调,“有无”选项无法精准反映班级基础层次。
            \item \textbf{产品反馈}:
                \begin{enumerate}
                    \item 学生名单导入错误,部分学生不在项目组织中,无法参与课堂小测;
                    \item 代码不能编辑 Markdown 单元格,否则会造成 CCI 假死,无法运行后续代码;
                    \item CCI 时长资源不足,时间不够需要在临内项群中手动续时。
                \end{enumerate}
        \end{itemize}
    \end{alertblock}
\end{frame}
